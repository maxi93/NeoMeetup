%%%%%%%%%%%%%%%%%%%%%%%%%%%%%%%%%%%%%%%%%
% Stylish Article
% LaTeX Template
% Version 2.1 (1/10/15)
%
% This template has been downloaded from:
% http://www.LaTeXTemplates.com
%
% Original author:
% Mathias Legrand (legrand.mathias@gmail.com) 
% With extensive modifications by:
% Vel (vel@latextemplates.com)
%
% License:
% CC BY-NC-SA 3.0 (http://creativecommons.org/licenses/by-nc-sa/3.0/)
%
%%%%%%%%%%%%%%%%%%%%%%%%%%%%%%%%%%%%%%%%%

%----------------------------------------------------------------------------------------
%	PACKAGES AND OTHER DOCUMENT CONFIGURATIONS
%----------------------------------------------------------------------------------------

\documentclass[fleqn,10pt]{SelfArx} % Document font size and equations flushed left

\usepackage[english]{babel} % Specify a different language here - english by default

\usepackage{lipsum} % Required to insert dummy text. To be removed otherwise

\usepackage[defaultfam,light,tabular,lining]{montserrat} %% Option 'defaultfam'
%% only if the base font of the document is to be sans serig
\usepackage[T1]{fontenc}
\renewcommand*\oldstylenums[1]{{\fontfamily{Montserrat-TOsF}\selectfont #1}}

%----------------------------------------------------------------------------------------
%	COLUMNS
%----------------------------------------------------------------------------------------

\setlength{\columnsep}{0.55cm} % Distance between the two columns of text
\setlength{\fboxrule}{0.75pt} % Width of the border around the abstract

%----------------------------------------------------------------------------------------
%	COLORS
%----------------------------------------------------------------------------------------

\definecolor{color1}{RGB}{102, 102, 102} % Color of the article title and sections
%\definecolor{color1}{RGB}{178,34,34} % Color of the article title and sections
%\definecolor{color1}{RGB}{246,64,96} % Color of the article title and sections

%\definecolor{color2}{RGB}{0,20,20} % Color of the boxes behind the abstract and headings
\definecolor{color2}{RGB}{246,64,96}
%----------------------------------------------------------------------------------------
%	HYPERLINKS
%----------------------------------------------------------------------------------------

\usepackage{hyperref} % Required for hyperlinks
\hypersetup{hidelinks,colorlinks,breaklinks=true,urlcolor=color2,citecolor=color1,linkcolor=color1,bookmarksopen=false,pdftitle={Title},pdfauthor={Author}}

%----------------------------------------------------------------------------------------
%	ARTICLE INFORMATION
%----------------------------------------------------------------------------------------

\JournalInfo{.} % Journal information
\Archive{ } % Additional notes (e.g. copyright, DOI, review/research article)

\PaperTitle{NeoMeetup: blabla} % Article title

\Authors{Dario Bertazioli\textsuperscript{1}, Fabrizio D'Intinosante\textsuperscript{1}, Massimiliano Perletti\textsuperscript{1}*} % Authors
\affiliation{\textsuperscript{1}\textit{Data Science, Department of Computer Science, University of Milano Bicocca, Milan, Italy}} % Author affiliation
\affiliation{*\textbf{Corresponding author}: m.perletti@campus.unimib.it} % Corresponding author

\Keywords{Meetup --- ADD KEYWORDS--- Keyword3} % Keywords - if you don't want any simply remove all the text between the curly brackets
\newcommand{\keywordname}{Keywords} % Defines the keywords heading name

%----------------------------------------------------------------------------------------
%	ABSTRACT
%----------------------------------------------------------------------------------------

\Abstract{yolo}

%----------------------------------------------------------------------------------------

\begin{document}

\flushbottom % Makes all text pages the same height

\maketitle % Print the title and abstract box

\tableofcontents % Print the contents section

\thispagestyle{empty} % Removes page numbering from the first page

%----------------------------------------------------------------------------------------
%	ARTICLE CONTENTS
%----------------------------------------------------------------------------------------

\section*{Introduction}
\textbf{Meetup} è stato creato nel 2002 come piattaforma per mettere in contatto le persone nella vita reale. Fondato e guidato inizialmente da Scott Heiferman e Brendan McGovern, nel 2017 Meetup è stato acquisito da WeWork (ora The We Company). \\
Una volta completata la fase di registrazione, gli utenti possono: \\
\\
- Selezionare i propri interessi: ciò avviene sottoscrivendo dei topic precompilati dall'applicazione in modo da favorire il sistema di raccomandazione. Questi topic spaziano tra i più svariati ambiti, da quello professionale a quello degli hobby, fino ai più comuni, relativi ad eventi sociali. \\
\\
- Iscriversi a gruppi locali: una volta selezionati i topic di interesse il sistema di raccomandazione dell'applicazione suggerisce all'utente una serie di gruppi più o meno locali (a seconda dell'area di interesse selezionata dall'utente) che trattano i topic in oggetto o altri topic ad essi correlati. Ciò permette agli utenti di incontrare persone della propria zona che condividono le stesse passioni. Nel caso in cui tra i topic di interesse per l'utente ce ne siano alcuni che non trovano nessun riscontro in gruppi presenti sul territorio, l'applicazione suggerisce all'utente di creare lui stesso un gruppo con oggetto quel topic, consigliando di mettersi in contatto con altre persone che dimostrano quell'interesse, suggerendole attraverso il sistema di raccomandazione.\\
\\
- Partecipare ad eventi: i gruppi locali, nel corso del tempo, organizzano eventi, incontri, meeting con oggetto i topic dichiarati dai gruppi stessi. Una volta che un gruppo ha organizzato un evento l'utente può visualizzarli sulla propria home e, una volta visionati i dettagli, decidere di comunicare se parteciperà oppure no e, nel caso volesse partecipare, se ha intenzione di portare degli ospiti, il tutto in maniera non vincolante. Inoltre, per non restringere troppo la sfera di interessi degli utenti, l'applicazione permette anche di selezionare diversi filtri per la home, tra cui uno apposito per visualizzare eventi organizzati da gruppi di cui non si fa parte o un filtro apposito per visualizzare gruppi di cui l'utente non è già membro e che magari non trattano topic per cui l'utente ha dichiarato interesse, ma che essendo pur sempre gruppi locali, l'utente potrebbe gradire o trovare interessanti.\\
\\
Il punto centrale dell'intero meccanismo dell'applicazione risulta quindi essere quello del "gruppo" visto come realtà associativa e fautore di momenti di aggregazione durante l'intero anno, molto spesso non guidati da una ristretta cerchia di capi ma, anzi, promotore di iniziative da parte dei suoi stessi membri. La varietà di gruppi, a seconda dei topic trattati, spazia tra:
\begin{enumerate}
\item Gruppi di socializzazione, che hanno come obiettivo principale quello di svolgere attività ludiche in compagnia, molto spesso anche promotori di veri e propri eventi di incontro per single.
\item Gruppi professionali, che si pongono l'obiettivo di mettere in contatto persone e professionisti di svariati campi attraverso workshop o presentazioni con il fine di fare crescere gli utenti dal punto di vista professionale.
\item Gruppi creativi, ovvero gruppi di progettazione e più vocati ad hobby ed alla pratica delle più svariate arti.
\end{enumerate}
Sintetizzando, quindi, la missione di Meetup è aiutare le persone a crescere e raggiungere i loro obiettivi attraverso connessioni autentiche e reali. Dal network professionale alla birra artigianale passando per workshop di programmazione, le persone usano Meetup per uscire dalla loro comfort zone, incontrare nuove persone, imparare cose nuove, perseguire le loro passioni e trovare sostegno in comunità che le aiutano a crescere.\\
Ad oggi, Meetup è disponibile in 186 Paesi, e conta più di 40 milioni di membri sulla piattaforma, con più di 320 mila gruppi attivi ed una media di 12 mila eventi al giorno.
%------------------------------------------------

\section{Goals}
Attraverso l'analisi di questa rete sociale e andando a studiare le relazioni che interconnettono le persone attraverso eventi sociali organizzati in tutto il globo, è possibile avere una mappatura degli interessi comuni che portano ad agglomerare un significativo numero di soggetti.
Ricreando quindi questa rete, l'obiettivo del progetto è quello di identificare correlazioni o tracce significative che possono determinare delle buone regole nella creazione, in primis di un gruppo sociale, %\non ho idea QUI se è vero
di eventi di notevole impatto sociale, che possano raggruppare un buon numero di persone.(CONTINUE)%first version incomplete  

%------------------------------------------------
\section{Implementation: Architecture}
\subsection{Source}: WebSocket and blabla
\subsection{Nifi}: Producer and blabla
\subsection{Kafka}: Lambda Arch. and blabla
\subsection{Neo4j}: Storing and querying

\section{Results}

Fabri  $<$3 Meetup xD.
%------------------------------------------------
\phantomsection
\section*{Interesting Challenges}
Circumstances where fabri has sclerated.
\section*{Acknowledgments} % The \section*{} command stops section numbering

\addcontentsline{toc}{section}{Acknowledgments} % Adds this section to the table of contents

So long and thanks for all the fish.

%----------------------------------------------------------------------------------------
%	REFERENCE LIST
%----------------------------------------------------------------------------------------
\phantomsection
\bibliographystyle{unsrt}
\bibliography{sample}

%----------------------------------------------------------------------------------------

\end{document}